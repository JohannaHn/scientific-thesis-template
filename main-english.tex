% !TeX spellcheck = en-US
% !TeX encoding = utf8
% !TeX program = pdflatex
% !BIB program = biber
% -*- coding:utf-8 mod:LaTeX -*-


% vv  scroll down to line 200 for content  vv


\let\ifdeutsch\iffalse
\let\ifenglisch\iftrue
\input{pre-documentclass}
\documentclass[
  % fontsize=11pt is the standard
  a4paper,  % Standard format - only KOMAScript uses paper=a4 - https://tex.stackexchange.com/a/61044/9075
  twoside,  % we are optimizing for both screen and two-sided printing. So the page numbers will jump, but the content is configured to stay in the middle (by using the geometry package)
  bibliography=totoc,
  %               idxtotoc,   %Index ins Inhaltsverzeichnis
  %               liststotoc, %List of X ins Inhaltsverzeichnis, mit liststotocnumbered werden die Abbildungsverzeichnisse nummeriert
  headsepline,
  cleardoublepage=empty,
  parskip=half,
  %               draft    % um zu sehen, wo noch nachgebessert werden muss - wichtig, da Bindungskorrektur mit drin
  draft=false
]{scrbook}
\input{config}


\usepackage[
  title={Is Oil the future?},
  author={Johanna Hofmann},
  type=bachelor,
  institute=iaas, % or other institute names - or just a plain string using {Demo\\Demo...}
  course={Medieninformatik},
  examiner={Prof.\ Dr.\ Uwe Fessor},
  supervisor={Dipl.-Inf.\ Roman Tiker,\\Dipl.-Inf.\ Laura Stern,\\Otto Normalverbraucher,\ M.Sc.},
  startdate={July 5, 2018},
  enddate={January 5, 2019}
]{scientific-thesis-cover}

\input{acronyms}

\makeindex

\begin{document}

%tex4ht-Konvertierung verschönern
\iftex4ht
  % tell tex4ht to create pictures also for formulas starting with '$'
  % WARNING: a tex4ht run now takes forever!
  \Configure{$}{\PicMath}{\EndPicMath}{}
  %$ % <- syntax highlighting fix for emacs
  \Css{body {text-align:justify;}}

  %conversion of .pdf to .png
  \Configure{graphics*}
  {pdf}
  {\Needs{"convert \csname Gin@base\endcsname.pdf
      \csname Gin@base\endcsname.png"}%
    \Picture[pict]{\csname Gin@base\endcsname.png}%
  }
\fi

%\VerbatimFootnotes %verbatim text in Fußnoten erlauben. Geht normalerweise nicht.

\input{commands}
\pagenumbering{arabic}
\Titelblatt

%Eigener Seitenstil fuer die Kurzfassung und das Inhaltsverzeichnis
\deftriplepagestyle{preamble}{}{}{}{}{}{\pagemark}
%Doku zu deftriplepagestyle: scrguide.pdf
\pagestyle{preamble}
\renewcommand*{\chapterpagestyle}{preamble}



%Kurzfassung / abstract
%auch im Stil vom Inhaltsverzeichnis
\ifdeutsch
  \section*{Kurzfassung}
\else
  \section*{Abstract}
\fi

<Short summary of the thesis>

\cleardoublepage


% BEGIN: Verzeichnisse

\iftex4ht
\else
  \microtypesetup{protrusion=false}
\fi

%%%
% Literaturverzeichnis ins TOC mit aufnehmen, aber nur wenn nichts anderes mehr hilft!
% \addcontentsline{toc}{chapter}{Literaturverzeichnis}
%
% oder zB
%\addcontentsline{toc}{section}{Abkürzungsverzeichnis}
%
%%%

%Produce table of contents
%
%In case you have trouble with headings reaching into the page numbers, enable the following three lines.
%Hint by http://golatex.de/inhaltsverzeichnis-schreibt-ueber-rand-t3106.html
%
%\makeatletter
%\renewcommand{\@pnumwidth}{2em}
%\makeatother
%
\tableofcontents

% Bei einem ungünstigen Seitenumbruch im Inhaltsverzeichnis, kann dieser mit
% \addtocontents{toc}{\protect\newpage}
% an der passenden Stelle im Fließtext erzwungen werden.

\listoffigures
\listoftables

%Wird nur bei Verwendung von der lstlisting-Umgebung mit dem "caption"-Parameter benoetigt
%\lstlistoflistings
%ansonsten:
\ifdeutsch
  \listof{Listing}{Verzeichnis der Listings}
\else
  \listof{Listing}{List of Listings}
\fi

%mittels \newfloat wurde die Algorithmus-Gleitumgebung definiert.
%Mit folgendem Befehl werden alle floats dieses Typs ausgegeben
\ifdeutsch
  \listof{Algorithmus}{Verzeichnis der Algorithmen}
\else
  \listof{Algorithmus}{List of Algorithms}
\fi
%\listofalgorithms %Ist nur für Algorithmen, die mittels \begin{algorithm} umschlossen werden, nötig

% Abkürzungsverzeichnis
\printnoidxglossaries

\iftex4ht
\else
  %Optischen Randausgleich und Grauwertkorrektur wieder aktivieren
  \microtypesetup{protrusion=true}
\fi

% END: Verzeichnisse


% Headline and footline
\renewcommand*{\chapterpagestyle}{scrplain}
\pagestyle{scrheadings}
\pagestyle{scrheadings}
\ihead[]{}
\chead[]{}
\ohead[]{\headmark}
\cfoot[]{}
\ofoot[\usekomafont{pagenumber}\thepage]{\usekomafont{pagenumber}\thepage}
\ifoot[]{}


%% vv  scroll down for content  vv %%































%%%%%%%%%%%%%%%%%%%%%%%%%%%%%%%%%%%%%%%%%%%%%%%%%%%%%%%%%%%%%%%%%%%%%%%%%%%%%%
%
% Main content starts here
%
%%%%%%%%%%%%%%%%%%%%%%%%%%%%%%%%%%%%%%%%%%%%%%%%%%%%%%%%%%%%%%%%%%%%%%%%%%%%%%


% !TeX spellcheck = de_DE

\chapter{Introduction}
\textcolor{red}{Different types of neural networks (NNs) exist for various types of data, and a choice must
be made based on the application. For instance, Convolutional neural networks (CNNs) are well
suited for modeling image/spatial data fields, whereas Recurrent neural networks (RNNs) are specialized NNs for modeling sequential data, like time signals. \cite{mohan2018deep}.}
\section*{Gliederung}
Die Arbeit ist in folgender Weise gegliedert:
\begin{description}
\item[Kapitel~\ref{chap:k2} -- \nameref{chap:k2}:] Hier werden werden die Grundlagen dieser Arbeit beschrieben.
\item[Kapitel~\ref{chap:zusfas} -- \nameref{chap:zusfas}] fasst die Ergebnisse der Arbeit zusammen und stellt Anknüpfungspunkte vor.
\end{description}




This thesis starts with \cref{chap:k2}.

We can also typeset \verb|<text>verbatim text</text>|.
Backticks are also rendered correctly: \verb|`words in backticks`|.

% !TeX spellcheck = de_DE

\chapter{Background}
\textcolor{red}{Machine learning is rapidly becoming a core technology for scientific computing, with numerous opportunities to advance the field of computational fluid dynamics. } \cite{vin2021}

As Vinuesa et al. \cite{vin2021} state, machine learning plays an increasing role in scientific computing and facilitates advances especially in the field of computational fluid dynamics. 
\section{Related Work}
Approaches for heat plume prediction with CNNs have been put forth by Leiteritz et al. \cite{leiteritz2022deep}, Davis et al. \cite{davis2023deep} and Pelzer and Schulte \cite{pelzer2024}. 
While \cite{leiteritz2022deep} and \cite{davis2023deep} use the subsurface velocity field as input, \cite{pelzer2024} use the pressure gradients and subsurface permeability.
In contrast to the subsurface velocity, the latter features can be measured directly at the bore hole, which improves the applicability of the model. Furthermore, \cite{pelzer2024} used a two stage approach: the first stage models single heat plumes neglecting the influence of surrounding heat pumps. Expanding on this, the second stage refines the predictions by taking into account interactions with adjacent heat plumes.

TODO - PINNs: The approach of using physics informed neural networks for 

In the field of turbulent flow control Mohan and Gaitonde \cite{mohan2018deep} made an advance in using LSTMs for fluid mechanic modeling. In their work, they developed a LSTM for predicting the key physical features of a high-fidelity flow field. 

\begin{itemize}
    \item The input signal covers 0-0.1 seconds and the prediction horizon is 0.1-0.2 seconds. 
    \item They also applied the \textbf{Hurst exponent} to assess the suitability of a LSTM as a predictive modelling approach.
    \item They employed a multiple model approach. One model for each POD mode.
\end{itemize}


This work aims on leveraging 

As LSTMs are predominantly used for sequence modeling 






\section{Neural Networks}
\chapter{Problem}
\label{chap:k2}

Hier wird der Hauptteil stehen. Falls mehrere Kapitel gewünscht, entweder mehrmals \texttt{\textbackslash{}chapter} benutzen oder pro Kapitel eine eigene Datei anlegen und \texttt{ausarbeitung.tex} anpassen.


\section{Data}
There are three data sets:

\begin{enumerate}
    \item dataset\_long\_k\_3e-10\_7dp
    \item dataset\_medium\_100dp\_vary\_perm
    \item dataset\_medium\_k\_3e-10\_1000dp
\end{enumerate}
\section{Goal}

\chapter{Method}
\begin{itemize}
    \item Predicting the heat plume of a GWHP can be viewed as a sequence modeling problem in machine learning.
    \item Problem Classification
    \begin{itemize}
        \item Sequence-to-Sequence Learning \cite{sutskever2014sequence}
        \item \textbf{Spatiotemporal sequence forecasting} problem in which both the input and the prediction target are spatiotemporal sequence \cite{ShiConvLSTMPrecipitation}
    \end{itemize}
    \item Why FC-LSTM does not work: \newline
    The major impediment of FC-LSTMs for spatiotemporal data is its insufficient ability to capture spatial information. 
    It is unclear how a machine learning algorithm could handel the spatiotemporal data.
    This issue is mitigated by using convolutional layers in the input-to-state and state-to-state-transitions. 
    The convolutional layers also supports the encoding-forecasting strucutre. 
\end{itemize}

\section{Network Architecture}

\section{Parameter, Functions etc.}
\section{Training Strategy}
% Alternativ: Literaturrecherche nach Introduction
%Problem Kapitel aufteilen

\chapter{Evaluation}
\section{Test Design}
\section{Results}

%noch etwas Fülltext
%\blinddocument


\printbibliography

All links were last followed on March 17, 2018.

\appendix
\input{latexhints-english}

\pagestyle{empty}
\renewcommand*{\chapterpagestyle}{empty}
\Versicherung
\end{document}
